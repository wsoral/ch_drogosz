\documentclass[man]{apa6}

\usepackage[utf8]{inputenc}
\usepackage[T1]{fontenc}
\usepackage[MeX]{polski}

\makeatletter
\renewcommand\efloat@iwrite[1]{%
   \immediate\expandafter\protected@write\csname efloat@post#1\endcsname{}}
\makeatother

\usepackage{csquotes}
\usepackage[style=apa,sortcites=true,sorting=nyt,backend=biber]{biblatex}
\DeclareLanguageMapping{english}{american-apa}
\addbibresource{biblio.bib}

\usepackage{geometry}
\geometry{letterpaper}
\usepackage{graphicx}


\title{Dlaczego ludzie szukają wyjaśnień spiskowych zdarzeń społecznych? Mechanizmy psycho-społeczne.}
\shorttitle{Dlaczego ludzie szukają wyjaśnień spiskowych}
\author{Mirosław Kofta, Wiktor Soral}
\affiliation{Uniwersytet Warszawski}
\abstract{Lorem ipsum dolor sit amet, consectetur adipisicing elit, sed do eiusmod tempor incididunt ut labore et dolore magna aliqua. Ut enim ad minim veniam, quis nostrud exercitation ullamco laboris nisi ut aliquip ex ea commodo consequat. Duis aute irure dolor in reprehenderit in voluptate velit esse cillum dolore eu fugiat nulla pariatur. Excepteur sint occaecat cupidatat non proident, sunt in culpa qui officia deserunt mollit anim id est laborum.}

\begin{document}
    \maketitle
    ``Bernie musi zostać zmielony na miazgę. Nie możemy zacząć wierzyć we własne głupie gadanie o zasadach.[...] Zgnieć go jak tylko potrafisz. Tak poza tym, mam nadzieję, że wszystko w porządku i gratulacje za Nevadę.'' Wyciek tego fragmentu korespondencji Johna Podesty -- szefa kampanii prezydenckiej Hillary Clinton -- wstrząsnął opinią publiczną nie tylko w Stanach Zjednoczonych, ale również na całym świecie. Ten skandaliczny fragment nie pozostawił złudzeń, pokazując o jaką stawkę toczy się polityczna gra, ale jednocześnie przypominając jak ogromna część tej gry jest ukryta przed opinią publiczną, i jak często nieczyste zagrania są zasłaniane uśmiechniętymi twarzami polityków. Ten fragment pozostawił niesmak, ale niewątpliwie wzbudził również nieufność oraz niepewność wśród elektoratu Clinton, w destrukcyjny sposób wpływając na przebieg jej kampanii prezydenckiej.\\

    Życie polityczne nieustannie dostarcza zdarzeń wzbudzających wysoki poziom nieufności, niepewności, bezsilności, poczucia braku sensu, oraz szeregu innych zagrożeń dla Ja. W czasach kryzysu i braku stabilności politycznej, ten zestaw różnych negatywnych stanów psychologicznych niezwykle często jest przyczyną powstawania i rozprzestrzeniania się naiwnych i upraszczających teorii. Te spiskowe teorie stanowią próbę wyjaśnienia przyczyn klęsk, katastrof i niepowodzeń, a jednocześnie wskazują winnych -- niewielkie grupy lub stowarzyszenia osób spotykających się potajemnie, i w tajemnicy knujących plany wprowadzenia chaosu w świecie społecznym. Przykład korespondencji Podesty pokazuje, że spiskowy obraz rzeczywistości bywa niekiedy niepokojąco zbliżony do realiów życia politycznego.\\

    W niniejszym rozdziale pragniemy wskazać w jaki sposób rzeczywistość polityczna oraz orientacja spiskowa (tj. skłonność do postrzegania świata społecznego jako rządzonego przez spiski) są przez siebie wzajemnie kształtowane.

    \section{Natura myślenia spiskowego. Teorie spiskowe zdarzeń i grup społecznych}

    Lorem ipsum dolor sit amet, consectetur adipisicing elit, sed do eiusmod tempor incididunt ut labore et dolore magna aliqua. Ut enim ad minim veniam, quis nostrud exercitation ullamco laboris nisi ut aliquip ex ea commodo consequat. Duis aute irure dolor in reprehenderit in voluptate velit esse cillum dolore eu fugiat nulla pariatur. Excepteur sint occaecat cupidatat non proident, sunt in culpa qui officia deserunt mollit anim id est laborum.

    \section{Nieprzewidywalny i groźny świat jako źródło myślenia spiskowego}

    Lorem ipsum dolor sit amet, consectetur adipisicing elit, sed do eiusmod tempor incididunt ut labore et dolore magna aliqua. Ut enim ad minim veniam, quis nostrud exercitation ullamco laboris nisi ut aliquip ex ea commodo consequat. Duis aute irure dolor in reprehenderit in voluptate velit esse cillum dolore eu fugiat nulla pariatur. Excepteur sint occaecat cupidatat non proident, sunt in culpa qui officia deserunt mollit anim id est laborum.

    \section{Zagrożenie podmiotowej kontroli a generowanie teorii spiskowych}

    Szereg dowodów wskazuje, że zagrożenie podmiotowej kontroli skutkuje zwiększoną skłonnością do wiary w przesądy oraz teorie spiskowe \parencite{whitson2008lacking}. Warto tutaj wskazać co najmniej dwa mechanizmy odpowiadające za tą zależność. Pierwszy z nich ma charakter motywacyjny, natomiast drugi wynika z poznawczych konsekwencji (deficytów) deprywacji kontroli.\\

    Narastające doświadczenie utraty kontroli nad biegiem zdarzeń sprzyja uruchomieniu mechanizmu kompensacyjnego, polegającego na poszukiwaniu zewnętrznego źródła (społecznej lub duchowej) sprawczości. Jak wykazały badania \textcite{kay2008god} deprywacja kontroli skłania do poszukiwania w świecie zewnętrznych, zdolnych do przywrócenia porządku oraz kontroli nad biegiem zdarzeń. Pod wpływem zagrożenia utratą kontroli osoby zaczynają w większym stopniu pokładać zaufanie w rządzie lub w Bogu

Aaron Kay and his colleagues found that deprivation of personal control makes people to identify external forces in the world that could restore control over the course of events.

Loosing personal control resulted in perceiving government as legitimate (Kay et al., 2008, Shepard, Kay, 2009) and god as controlling force (Laurin, Kay, Moscovitsch, 2008).

Other findings suggest that control deprivation may turn our attention not only toward ``benevolent'' collective and spiritual agents (as results of Kay and colleagues seem to suggest), but also toward ``malevolent'' collective agents.

Sullivan, Landau, and Rothshild  (2010) have proposed that, when confronted with multiple uncontrollable threatening events, people tend to generate an idea of personal or collective enemy, constructed as a single source of evil. This process helps people to regain control over the word: once a single source of evil is identified, people could be more likely to blame the enemy and initiate actions intended to its elimination. Thus, according to Sullivan’s at al. theorizing and findings, loss of control may contribute to generation of negative collective agents, i.e., groups conspiring against ``us''.


    Poczucie braku podmiotowej kontroli skłania do poszukiwania sił sprawujących kontrolę nad rzeczywistością.

    Poczucie braku podmiotowej kontroli skłania do poszukiwania uproszczonych wyjaśnień rzeczywistości społecznej.

    \section{Rzeczywistość polityczna jako generator orientacji spiskowej -- badania własne}

    W niniejszym rozdziale wykazaliśmy, że w ramach rzeczywistości politycznej niezwykle często dochodzi do zdarzeń, które mogą stanowić zagrożenie dla realizacji potrzeb bezpieczeństwa, poczucia pewności i przewidywalności, oraz potrzeby podmiotowej kontroli nad biegiem zdarzeń. Aby radzić sobie z tą zagrażającą rzeczywistością, osoby w niej funkcjonujące mogą chętniej przyjmować teorie spiskowe wyjaśniające przyczyny: katastrof, kryzysów politycznych lub ekonomicznych. Przyjęcie tych teorii pozwala bowiem na -- przynajmniej chwilowe -- uzyskanie pewności oraz  poczucia kontroli. Stąd też można powiedzieć, że cechą rzeczywistości politycznej jest tendencja do kształtowania przez nią wśród osób w niej funkcjonujących względnie stałej orientacji spiskowej.\\
    Przedstawimy teraz szereg własnych badań przeprowadzonych w Polsce w jej obecnym systemie politycznym, których wnioski stanowią cząstkowe argumenty przemawiające za naszą hipotezą. Omówimy badania, które wskazują na centralne miejsce teorii spiskowych w konflikcie politycznym i mogą stanowić narzędzie mobilizacji politycznej wśród elektoratów partii. W dalszej części, kontynuując wątek teorii spiskowych jako narzędzi walki politycznej, przedstawimy w jaki sposób wolne wybory mogą skłaniać do przyjmowania teorii spiskowych. W końcu wskażemy, że teorie spiskowe mogą pełnić funkcję nie tylko mobilizacyjną, ale również mogą być ściśle związane z poczuciem brak wpływu na rzeczywistość polityczną. Wiara w teorie spiskowe może bowiem wynikać z poczucia bezsilności politycznej, ale również samo przyjmowanie teorii spiskowych może prowadzić do wzrostu poczucia bezsilności jednostki wobec systemu politycznego.


	\subsection{Teoria spiskowa jako narzędzie mobilizacji i walki politycznej}

	Obecnie dość dobrze wiadomo, że u podłoża wielu teorii spiskowych leży brak zaufania w stosunku do elit politycznych oraz rządów państwowych \parencite[zob. przegląd w,][]{van2014power}. Stąd też różne przekonania spiskowe mogą ze sobą pozytywnie korelować, nawet jeżeli z logicznego punktu widzenia wzajemnie się wykluczają. Przykładowo, \textcite{wood2012dead} w efektownych badaniach wykazali, że osoby przekonane, że Osama bin Laden był już martwy w chwili ataku amerykańskich sił specjalnych na jego kryjówkę w Pakistanie, istotnie częściej były przekonane również co do tego, że Osama bin Laden nadal żyje. Co istotne ta obserwowana korelacja znikała jeżeli w modelu uwzględniono siłę przekonanie, że władze państwowe są uwikłane w spisek mający na celu ukrycie istotnych informacji odnośnie działań tajnych służb. Można więc powiedzieć, że tym co sprzyja przyjmowaniu teorii spiskowych jest bycie w opozycji do władzy lub na ekstremalnych krańcach spektrum politycznego \parencite[patrz, np.,][]{inglehart1987extremist}.\\
	W takim rozumieniu wiarę w teorie spiskowe można rozumieć jako przejaw zagrożenia ze strony nie tylko obcych narodów, ale również ze strony wrogich ugrupowań politycznych, w tym również tych ugrupowań sprawujących władzę państwową. Co więcej, o przyjmowaniu spiskowych wyjaśnień zdarzeń społecznych decydować będzie nie tyle wymiar prawicowości vs. lewicowości, co raczej nieufność i podejrzliwość wobec grup postrzeganych jako żywiące wrogie intencje w stosunku do grupy własnej \parencite{prooijen2015mutual}. Empirycznych dowodów na rzecz tej tezy dostarczają badania \textcite{golec2012collective}, w których wykazano że osoby o wysokim poziomie kolektywnego narcyzmu w większym stopniu wierzą w antysemickie teorie spiskowe. Zdaniem autorek cytowanych badań, osoby o wysokim poziomie kolektywnego narcyzmu są szczególnie wyczulone na wszelkie zagrożenia, których źródłem są grupy obce. Relację pomiędzy kolektywnym narcyzmem a antysemityzmem spiskowym można więc zrozumieć jeżeli przyjąć, że u podstaw stereotypów spiskowych, i bardziej ogólnie teorii spiskowych, leży podejrzliwość, nieufność, lęk przed skrywanymi działaniami wrogich ugrupowań.\\
	W świetle tych wszystkich dowodów, zrozumiałe jest dlaczego ugrupowania polityczne tak często odwołują się do teorii spiskowych jako narzędzia walki politycznej. W walce politycznej, rolą teorii spiskowych jest bowiem stałe podkopywanie zaufania w stosunku do przeciwnika politycznego zarówno wśród partyjnego elektoratu, jak i wśród tych, którzy do elektoratu nie należą, ale z różnych względów nie są również zwolennikami przeciwnika politycznego. W związku z tym warto przyjrzeć się roli jaką mogły pełnić różne teorie wskazujące na spiskowe wyjaśnienia katastrofy samolotu prezydenckiego pod Smoleńskiem z dnia 10. kwietnia 2010 roku.\\
    Według danych zebranych w ramach Polskiego Sondażu Uprzedzeń, trzy lata po katastrofie, w 2013 wyjaśnienie katastrofy smoleńskiego nadal było tematem niezwykle spornym. Według sondażu \parencite[patrz,][]{soral2015socjo} ok. 25 procent Polaków było przekonanych, że przyczyną katastrofy smoleńskiej był zamach. Około 53 procent Polaków uważało, że katastrofa była wynikiem wypadku, a prawie 61 procent Polaków wskazywało że katastrofa smoleńska była efektem bałaganu i złej organizacji. Warto zaznaczyć, że niezależnie od przekonań co do przyczyn katastrofy, niemal 46 procent Polaków uważało, że władze polskie oraz rosyjskie wspólnie zatajają prawdę na temat jej przyczyn.\\
    Re-analizy danych z omawianego sondażu wykazały, że stopień wiary w spiskowe wyjaśnienia katastrofy smoleńskiej korelował pozytywnie z teorią spiskową dotyczącą Żydów (patrz Rycina \ref{fig:fig1}a). Ciekawe jest, że choć obie teorie dotyczyły zupełnie różnych zdarzeń społecznych, w percepcji uczestników sondażu wydawały się być ze sobą powiązane. Należy zaznaczyć, że zarówno wiara w spisek smoleński jak i wiara w spisek żydowski były pozytywnie związane z przekonaniem, że władze państwowe zatajają ważne informacje przed swoimi obywatelami (patrz Rycina \ref{fig:fig1}b). Co istotne, przy kontroli tej trzeciej zmiennej związek pomiędzy wiarą w spisek smoleński i wiarą w spisek żydowski niemal kompletnie zanikał.\\

	\begin{figure*}[htbp]
   		\centering
   		\fitfigure{figures/fig1.pdf}
   		\caption{Związek pomiędzy wiarą w spisek smoleński a antysemityzmem spiskowym, bez -- a -- i przy kontroli -- b -- poziomu przekonania, że władze zatajają informacje przed obywatelami. \\ *$p$ < 0,05 **$p$ < 0,01 **$p$ < 0,001}
   		\label{fig:fig1}
	\end{figure*}

	Biorąc pod uwagę te wyniki można postulować, że wiara w spisek smoleński, tak jak inne teorie spiskowe, jest napędzana nieufnością w stosunku do władz państwowych. Można również postulować, że przekonanie iż w Smoleńsku doszło do zamachu będzie charakterystyczne przede wszystkim dla osób wyczulonych na wszelkie przejawy zagrożenia ze strony grup obcych, czyli takie o wysokim poziomie kolektywnego narcyzmu. Niewątpliwie przekonanie, że władze polskie wspólnie z władzami rosyjskim w tajemnicy zaplanowały zamach przeciwko polskiemu prezydentowi stanowi informację szczególnie zagrażającą dla grupy własnej.\\
	Wyniki badań własnych nad psychologicznymi determinantami wiary w spisek smoleński potwierdzają hipotezy o szczególnej roli kolektywnego narcyzmu jako predyktora orientacji spiskowej, ale wskazują również na bardziej złożony charakter tej relacji.

	\ref{fig:fig2}

	\begin{figure*}[htbp]
   		\centering
   		\fitfigure{figures/fig2.pdf}
   		\caption{Kolektywny narcyzm a prawdopodobieństwo wiary w spisek smoleński wśród osób o różnych nasileniach potrzeby poznawczego domknięcia (NFC).}
   		\label{fig:fig2}
	\end{figure*}


    \subsection{Mobilizacja, alienacja polityczna a wiara w spiski}

    W badaniu na próbie internautów (N=812) pytaliśmy o poczucie braku wpływu na politykę, o poczucie braku zrozumienia polityki, oraz o wiarę w stereotyp spiskowy Żydów.


    	\ref{fig:fig3}

	\begin{figure*}[htbp]
   		\centering
   		\fitfigure{figures/fig3.pdf}
   		\caption{Relacja pomiędzy poczuciem politycznej bezsilność a antysemityzmem spiskowym w badaniu podłużnym z dwoma pomiarami (T1 i T2). Wartości na diagramie oznaczają współczynniki standaryzowane. \\
        *$p$ < 0,05 **$p$ < 0,01 ***$p$ < 0,001}
   		\label{fig:fig3}
	\end{figure*}


    \section{Podsumowanie}


\printbibliography

\end{document}
