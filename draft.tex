\documentclass[man]{apa6}

\usepackage[utf8]{inputenc}
\usepackage[T1]{fontenc}
\usepackage[MeX]{polski}

\makeatletter
\renewcommand\efloat@iwrite[1]{%
   \immediate\expandafter\protected@write\csname efloat@post#1\endcsname{}}
\makeatother

\usepackage{geometry}
\geometry{letterpaper}
\usepackage{graphicx}


\title{Dlaczego ludzie szukają wyjaśnień spiskowych zdarzeń społecznych? Mechanizmy psycho-społeczne.}
\shorttitle{Dlaczego ludzie szukają wyjaśnień spiskowych}
\author{Mirosław Kofta, Wiktor Soral}
\affiliation{Uniwersytet Warszawski}
\abstract{Lorem ipsum dolor sit amet, consectetur adipisicing elit, sed do eiusmod tempor incididunt ut labore et dolore magna aliqua. Ut enim ad minim veniam, quis nostrud exercitation ullamco laboris nisi ut aliquip ex ea commodo consequat. Duis aute irure dolor in reprehenderit in voluptate velit esse cillum dolore eu fugiat nulla pariatur. Excepteur sint occaecat cupidatat non proident, sunt in culpa qui officia deserunt mollit anim id est laborum.}

\begin{document}
    \maketitle

    ``Bernie musi zostać zmielony na miazgę. Nie możemy zacząć wierzyć we własne głupie gadanie o zasadach.[...] Zgnieć go jak tylko potrafisz. Tak poza tym, mam nadzieję, że wszystko w porządku i gratulacje za Nevadę.'' Wyciek tego fragmentu korespondencji Johna Podesty -- szefa kampanii prezydenckiej Hillary Clinton -- wstrząsnął opinią publiczną nie tylko w Stanach Zjednoczonych, ale również na całym świecie.

    Lorem ipsum dolor sit amet, consectetur adipisicing elit, sed do eiusmod tempor incididunt ut labore et dolore magna aliqua. Ut enim ad minim veniam, quis nostrud exercitation ullamco laboris nisi ut aliquip ex ea commodo consequat. Duis aute irure dolor in reprehenderit in voluptate velit esse cillum dolore eu fugiat nulla pariatur. Excepteur sint occaecat cupidatat non proident, sunt in culpa qui officia deserunt mollit anim id est laborum.

    \section{Natura myślenia spiskowego. Teorie spiskowe zdarzeń i grup społecznych}

    Lorem ipsum dolor sit amet, consectetur adipisicing elit, sed do eiusmod tempor incididunt ut labore et dolore magna aliqua. Ut enim ad minim veniam, quis nostrud exercitation ullamco laboris nisi ut aliquip ex ea commodo consequat. Duis aute irure dolor in reprehenderit in voluptate velit esse cillum dolore eu fugiat nulla pariatur. Excepteur sint occaecat cupidatat non proident, sunt in culpa qui officia deserunt mollit anim id est laborum.

    \section{Nieprzewidywalny i groźny świat jako źródło myślenia spiskowego}

    Lorem ipsum dolor sit amet, consectetur adipisicing elit, sed do eiusmod tempor incididunt ut labore et dolore magna aliqua. Ut enim ad minim veniam, quis nostrud exercitation ullamco laboris nisi ut aliquip ex ea commodo consequat. Duis aute irure dolor in reprehenderit in voluptate velit esse cillum dolore eu fugiat nulla pariatur. Excepteur sint occaecat cupidatat non proident, sunt in culpa qui officia deserunt mollit anim id est laborum.

    \section{Zagrożenie podmiotowej kontroli a generowanie teorii spiskowych}

    Poczucie braku podmiotowej kontroli skłania do poszukiwania sił sprawujących kontrolę nad rzeczywistością.

    Poczucie braku podmiotowej kontroli skłania do poszukiwania uproszczonych wyjaśnień rzeczywistości społecznej.

    \section{Rzeczywistość polityczna jako generator orientacji spiskowej -- badania własne}




    \subsection{Wolne wybory, niepewność i brak kontroli a teorie spiskowe}


    \subsection{Alienacja polityczna a wiara w spiski}

    W badaniu na próbie internautów (N=812) pytaliśmy o poczucie braku wpływu na politykę, o poczucie braku zrozumienia polityki, oraz o wiarę w stereotyp spiskowy Żydów.


    \subsection{Teoria spiskowa jako narzędzie mobilizacji i walki politycznej}



    W 2013 roku ok. 25 procent Polaków było przekonanych, że przyczyną katastrofy smoleńskiej był zamach. Około 53 procent Polaków uważało, że katastrofa była wynikiem wypadku, a prawie 61 procent Polaków wskazywało że katastrofa smoleńska była efektem bałaganu i złej organizacji. Warto zaznaczyć, że niezależnie od przekonań co do przyczyn katastrofy niemal 46 procent Polaków uważało, że władze polskie oraz rosyjskie wspólnie zatajają prawdę na temat jej przyczyn.

    Jak wykazali Soral i Grzesiak-Feldman (2015), spiskowe przekonania na temat spisku smoleńskiego mogą być charakterystyczne nie tylko dla osób o radykalnie prawicowych poglądach, ale raczej ogólnie dla osób na krańcach politycznego spektrum. Co więcej wskazali oni, że posiadanie wyrazistych poglądów politycznych, może skutkować chęcią dystansowania się od osób, które nie podzielają własnej ideologii.


    \section{Podsumowanie}



\end{document}
